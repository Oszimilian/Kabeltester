\section{Leistungsaufnahme}

\subsection{Ruheleistung}
Für die Berechnung der Strom / Leistungsaufnahme auf der Sekundärseite, wurden aus den Datenblättern der verbauten IC's die typische nominale Stromaufnahme übernommen. Dabei kam ich auf einen Wert von 10,23mA. Hinzu kommt der gemessene Ruhestrom der Konstantstromquelle von ca. 800uA. 
\\
Der Gesamtstromfluss der Sekundärseite liegt dementsprechend bei ca 11,03mA.
\\
Die Stromaufnahme der Primärseite liegt durch den Verpolungsschutz, der LED, dem INA und dem Spannungsteiler bei\\ etwa 2,61mA. 
\\
Bei der Berechnung der gesamten Ruheleistung wurde mit einem Wirkungsgrad des DC/DC-Wandlers von 98 Prozent gerechnet. 
\\
\\
Primärleistung: $P_{Primär} = 9,46V * 2,61mA = 24,5mW$
\\
\\
Sekundärleistung: $P_{Sekundär} = 4,91V * 11,03mA = 54,1mW$
\\
\\
Gesamtleistung: $P_{G_{Ruhestrom}} = P_{Primär} + \dfrac{P_{Sekundär}}{\eta} = 24,5mW + \dfrac{54,1mW}{0,98} = 79,7mW$
\\
\\
Bei der Inbetriebnahme wurde eine Leistungsaufnahme von ca. 100mW gemessen. Das entspricht einem vernachlässigbaren Delta von ca. 20mW. Dies kann durch die Messung, sowie durch die Bauteiltoleranzen auftreten und ist somit vernachlässigbar.

\subsection{Messleistung}

Der Messstrom während der Durchgangs- und Kurzschlussmessung ist zu vernachlässigen. Die Konstantstrommessung zieht auf der Sekundärseite für 4 Impulse mit einer Dauer von jeweils ca. 108ms einen Strom von 30mA.
\\
\\
Messleistung Sekundär: $P_{Messtrom_{Sekundär}} = +4,91V * I_{Messstrom} = 4,91V * 30mA = 147mW$
\\
\\
Messleistung Primär: $P_{Messstrom_{Primär}} = \dfrac{P_{Messtrom_{Sekundär}}}{\eta} = \dfrac{147mW}{0,98} = 150mW$


\subsection{Elektrische Arbeit}

Elektrische Arbeit für einen Messvorgang mit 4 Messleitungen.
\\
\\
$W_{Messvorgang} = (P_{G_{Ruhestrom}} + \dfrac{P_{Messstrom_{Primär}}}{(1 - t_{g})^{-1}}) * t_{Messzeit}$
\\
\\
$W_{Messvorgang} = (79,7mW + \dfrac{150mW}{(1-0,63)^{-1}}) * (\frac{912ms}{3600})$
\\
\\
$W_{Messvorgang} = 34,3uWh$

\subsection{Mögliche Messzyklen}
Es wurde mit einer Typischen Kapazität von 4,5 Wh des 9V Block gerechnet.
\\
\\
$Messzyklen = \frac{4,5Wh}{34,3 * 10^{-6}Wh} = 131195_{Messzyklen}$
\\
\\
Aufgrund der geringen Stromaufnahme und der großen Anzahl an Messzyklen wird auf die Verwendung von handelsüblichen 9V Batterien gesetzt. 
\\

\renewcommand{\arraystretch}{2}
\begin{tabularx}{\textwidth}{p{0.6\textwidth}| p{0.2\textwidth} | p{0.2\textwidth}}
Anzahl der LED's mit  $ON_{Time} = 1$			&	7 Stück		&	1mA 		\\
\hline
Anzahl der LED's mit  $ON_{Time} = 0,5	$	&	2 Stück		&	0,5mA 		\\
\hline
74HC08PW								&	4 Stück		&	2 uA		\\
\hline
74HC02PW								&	3 Stück		&	2 uA		\\
\hline
CD74HC4514M96							&	1 Stück		&	8 uA		\\
\hline
SN74LS93								&	1 Stück		&	9 uA		\\
\hline
TPS54331								&	1 Stück		&	110 uA		\\
\hline
74AHC1G14								&	3 Stück		&	1 uA		\\
\hline
74AHC2G08								&	2 Stück		&	1 uA		\\
\hline
NE555									&	1 Stück		&	2000 uA		\\
\hline
74AHC1G32								&	2 Stück		&	1 uA		\\
\hline
LTC6993									&	1 Stück		&	65 uA		\\
\hline
74LVC1G175								&	2 Stück		&	0,1 uA		\\
\hline
74LVC2G14								&	1 Stück		&	0,1 uA		\\
\hline
LMV321AUIDBVR							&	1 Stück		&	70 uA		\\
\hline
LMV358MM								&	1 Stück		&	130 uA		\\
\hline
74HC595									&	2 Stück		&	8 uA		\\
\hline
74HC1G126								&	1 Stück		&	1 uA		\\
\hline
74HC86PW								&	1 Stück		&	0,1 uA		\\
\hline
DG2535									&	2 Stück		&	1 uA		\\
\hline 
Gesamt									&				&	\textbf{10,23mA} \\
\end{tabularx}
\renewcommand{\arraystretch}{1}

%%Hallo Hallo
