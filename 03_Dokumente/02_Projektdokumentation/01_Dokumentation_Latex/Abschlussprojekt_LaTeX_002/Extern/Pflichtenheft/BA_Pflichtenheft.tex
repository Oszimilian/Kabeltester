\documentclass[a4paper,11pt]{scrartcl}

\usepackage[utf8]{inputenc}
\usepackage[ngerman]{babel}
\usepackage[T1]{fontenc}
\usepackage{amsmath}
\usepackage{graphicx}
\usepackage{tabularx}
\usepackage[a4paper, left=2cm, right=2cm, top=2.5cm, bottom=2.5cm]{geometry}
\usepackage{tikz}   
%\usetikzlibrary{calc}

%defaultfam

%\usepackage[light]{montserrat}
%\usepackage{libertine}
\usepackage[scaled]{helvet}
%\usepackage[scaled]{uarial}

\renewcommand*{\familydefault}{\sfdefault}

\usepackage{tabto} 

\usepackage{fancyhdr}
\pagestyle{fancy}
\lhead{Maximilian Hoffmann}
\chead{Betrieblicher Auftrag \\ \textbf{Kabeltester}}
\rhead{\includegraphics[width=3cm]{Bilder/BMK_LOGO.png}}


\begin{document}

\begin{center}
	\begin{huge}
	\textcolor{white}{\tiny{Platzhalter wenn ein Kapitel auf einer neuen Seite beginnt\\}}
	\textbf{Pflichtenheft}
	\end{huge}
\end{center}


\textbf{Auftraggeber:}
Herr Robert Schulz, BMK-Entwicklung

\section{Zielsetzung}

\begin{itemize}
	\item{Entwicklung eines Prototypen zur Prüfung von Flachbandkabel und Programmierkabel.}
\end{itemize}

\section{Technische Umsetzung}

\begin{itemize}
	\item{Kurzschlussprüfung aller Leitungen untereinander.}
	
	\item{Durchgangsprüfung jeder einzelnen Ader.}
	
	\item{Indirekte Prüfung des Ohmschen Widerstandes durch einen konstanten Stromfluss durch jede einzelne Ader.}
	
	\item{Separate Einstellung des Messstromes und dem Grenzwert des maximal zulässigen Leitungswiderstandes}
	
	\item{Maximale Anzahl der parallelen Leitungen pro Messvorgang des Prototypen ist auf 4 begrenzt.}
	
	\item{Optisches Feedback des Messergebnis jeder einzelnen Ader, durch rot oder grün leuchtende LED's.}
	
	\item{Versorgung der Schaltung durch eine 9V Batterie.}
	
	\item{Erweiterung der zu testenden Leitungen, durch das Verbinden zweier Messeinheiten.}
	
	\item{Einfaches Umschalten der Messeinheit zwischen Modus zum Verbinden zweier Messeinheiten und einfacher Modus.}
\end{itemize}

\section{Sonstige Anforderungen}

\begin{itemize}
	\item{Theoretische Analyse der maximal zulässigen Leitungswiderstand mit Hilfe des Simulationstool LTSpice.}
	
	\item{Berechnung der Leistungsaufnahme im Ruhe -und Messbetrieb und Validierung einer passenden portablen Versorgung.}
	
	\item{Erstellen einer Bedienungsanleitung.}
\end{itemize}

\section{Pflichten}

\begin{itemize}
	\item{Übergabe eines Prototypen bis um 14.5.2021 und einer projektspezifischen Dokumentation.}
\end{itemize}

\vspace{1,5cm}
\begin{tabularx}{\textwidth}[b]{ p{5cm} } \cline{1-1} 
Ort, Datum
\end{tabularx}

\vspace{1,5cm}
\begin{tabularx}{\textwidth}[b]{ p{5cm} X p{5cm} } \cline{1-1} \cline{3-3} 
Unterschrift Auftraggeber & & Unterschrift Auftragnehmer
\end{tabularx}

\end{document}