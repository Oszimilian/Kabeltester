\documentclass[a4paper,11pt]{scrartcl}

\usepackage[utf8]{inputenc}
\usepackage[ngerman]{babel}
\usepackage[T1]{fontenc}
\usepackage{amsmath}
\usepackage{graphicx}
\usepackage{tabularx}
\usepackage[a4paper, left=2cm, right=2cm, top=2.8cm, bottom=2.8cm]{geometry}
\usepackage{tikz}   
\usepackage[scaled]{helvet}
\usepackage{tabto} 
\usepackage{fancyhdr}
\usepackage{multirow}

\renewcommand*{\familydefault}{\sfdefault}

\pagestyle{fancy}

\setkomafont{section}{\huge}
\setkomafont{subsection}{\Large}


\lhead{Maximilian Hoffmann}
\chead{Betrieblicher Auftrag \\ \textbf{Kabeltester}}
\rhead{\includegraphics[width=3cm]{Bilder/BMK_LOGO.png}}

\begin{document}

\begin{center}
	\begin{huge}
	\textcolor{white}{\tiny{Platzhalter wenn ein Kapitel auf einer neuen Seite beginnt\\}}
	\textbf{Teilauftrag Layout}
	\end{huge}
\end{center}

\textbf{Auftraggeber}	\\
Maximilian Hoffmann		\\
BMK Group GmbH Co.KG	\\
Entwicklung Labor		\\
\\

\textbf{Auftragsnehmer}	\\
Angelo Ostermair		\\
BMK Group GmbH Co.KG	\\
Entwicklung Labor		\\

\section{Beschreibung}
Erstellen eines Layouts anhand des vorliegenden Schaltplanes in „KiCad“. Den Schaltplansymbolen liegt ein passendes Footprintsymbol bei.

\section{Anforderungen}

\begin{itemize}
	\item{4 Layer Design.}
	
	\item{Leiterplattengröße: 100 x 100 mm.}
	
	\item{Datenblatt spezifisches Layout des Schaltregler.}
	
	\item{Layer 2 und 3 sind mit \glqq GND \grqq{} und \glqq +5V \grqq{} Flächen zu füllen.}
	
	\item{Gegenüberliegende Platzierung von J1 und J2 (DSUB-Stecker)}
	
	\item{Erstellen von Gerberdateien: Signallayero; F B.Paste;	F B.Mask; Edge.Cuts}
	
	\item{KiCad-Version: 5.99-9156 (nightly Version)}
\end{itemize}

\section{Anlagen}

\begin{itemize}
	\item{KiCad-Projectfolder}
\end{itemize}

\vspace{1,5cm}
\begin{tabularx}{\textwidth}[b]{ p{5cm} } \cline{1-1} 
Ort, Datum
\end{tabularx}

\vspace{1,5cm}
\begin{tabularx}{\textwidth}[b]{ p{5cm} X p{5cm} } \cline{1-1} \cline{3-3} 
Unterschrift Auftraggeber & & Unterschrift Auftragnehmer
\end{tabularx}


\end{document}