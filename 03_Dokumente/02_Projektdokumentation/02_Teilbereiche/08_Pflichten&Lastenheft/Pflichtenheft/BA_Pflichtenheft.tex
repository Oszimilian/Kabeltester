\documentclass[a4paper,11pt]{scrartcl}

\usepackage[utf8]{inputenc}
\usepackage[ngerman]{babel}
\usepackage[T1]{fontenc}
\usepackage{amsmath}
\usepackage{graphicx}
\usepackage{tabularx}
\usepackage[a4paper, left=2cm, right=2cm, top=2.8cm, bottom=2.8cm]{geometry}
\usepackage{tikz}   
\usepackage[scaled]{helvet}
\usepackage{tabto} 
\usepackage{fancyhdr}
\usepackage{multirow}

\renewcommand*{\familydefault}{\sfdefault}

\pagestyle{fancy}

\setkomafont{section}{\huge}
\setkomafont{subsection}{\Large}


\lhead{Maximilian Hoffmann}
\chead{Betrieblicher Auftrag \\ \textbf{Kabeltester}}
\rhead{\includegraphics[width=3cm]{Bilder/BMK_LOGO.png}}



\begin{document}

\begin{center}
	\begin{huge}
	\textcolor{white}{\tiny{Platzhalter wenn ein Kapitel auf einer neuen Seite beginnt\\}}
	\textbf{Pflichtenheft}
	\end{huge}
\end{center}


\textbf{Auftraggeber:}
Herr Robert Schulz, BMK-Entwicklung

\section{Technische Anforderungen}

Der zu entwickelnde Prototyp soll folgende Funktionen haben:

\begin{itemize}
	\item{Möglichst verlustfreie Wandlung der Klemmspannung auf 5V TTL-Pegel.}
	
	\item{Kurzschluss -und Durchgangsprüfung.}
	
	\item{Indirekte Messung des Leitungswiderstand.}
	
	\item{Anzahl der Messlinien auf Prototyp: 4 Stück.}
	
	\item{Optisches Feedback des Messergebnis jeder einzelnen Ader.}
	
	\item{Einfaches verbinden mehrerer Messplatinen.}
	
	\item{Diskreter Aufbau mit Logik-IC's.}
\end{itemize}

\section{Pflichten}

\begin{itemize}
	\item{Bis zum 14.5.2021 wird dem Auftraggeber ein fertiger Prototyp mit einer voll umfänglichen Dokumentation übergeben.}
	
	\item{In einem Übergabegespräch werden dem Auftraggeber alle Ergebnisse der Entwicklungsarbeit mitgeteilt.}
	
	\item{Der Verlauf der Übergabe sowie die Ergebnisse werden schriftlich in einem Übergabeprotokoll festgehalten.}
\end{itemize}

\vspace{3cm}
\begin{tabularx}{\textwidth}[b]{ p{5cm} } \cline{1-1} 
Ort, Datum
\end{tabularx}

\vspace{1,5cm}
\begin{tabularx}{\textwidth}[b]{ p{5cm} X p{5cm} } \cline{1-1} \cline{3-3} 
Unterschrift Auftraggeber & & Unterschrift Auftragnehmer
\end{tabularx}

\end{document}