\documentclass[a4paper,11pt]{scrartcl}

\usepackage[utf8]{inputenc}
\usepackage[ngerman]{babel}
\usepackage[T1]{fontenc}
\usepackage{amsmath}
\usepackage{graphicx}
\usepackage{tabularx}
\usepackage[a4paper, left=2cm, right=2cm, top=3cm, bottom=3cm]{geometry}
\usepackage{tikz}   
%\usetikzlibrary{calc}

%defaultfam

%\usepackage[light]{montserrat}


\usepackage[scaled]{helvet}

\usepackage{tabto} 

\usepackage{fancyhdr}
\pagestyle{fancy}
\lhead{Maximilian Hoffmann}
\chead{Betrieblicher Auftrag \\ \textbf{Kabeltester}}
\rhead{\includegraphics[width=3cm]{Bilder/BMK_LOGO.png}}



\begin{document}



\begin{center}
	\begin{huge}
	\textbf{Lastenheft}
	\end{huge}
\end{center}

\textbf{Auftraggeber:}
Herr Robert Schulz, BMK-Entwicklung


\section{Zielsetzung}

\begin{itemize}
	\item{Entwicklung eines Prüfgerätes für Flachbandkabel und BMK-Interne standardisierter Programmierkabel.}	
\end{itemize}

\section{Anwendungsgebiet}

\begin{itemize}
	\item{BMK-Testentwicklungslabor: Prüfung der Flachbandkabel, welche Adapter-Kassette und Nadelbrett miteinander Verbinden.}
	
	\item{BMK-Entwicklung: Prüfung standardisierter Programmierkabel, für STMicroelectronics und Texas Instrument Mikrocontroller.}
\end{itemize}

\section{Technische Anforderungen}

\begin{itemize}
	\item{Kurzschluss- und Durchgangsprüfung.}
	\item{Prüfung auf Ohmschen Widerstand der Adern.}
	\item{Minimale Anzahl der parallelen Leitungen pro Messvorgang: 10.}
	\item{Versorgung durch Akku oder Batteriebetrieb.}
	\item{Einfache Erweiterung der zu testenden Leituntgen pro Kabel.}
\end{itemize}

\section{Sonstige Anforderungen}

\begin{itemize}
	\item{Analyse vom Grenzwert des maximal zulässigen Leitungswiderstandes.}
	\item{Lebenszeitberechnung im Akku/ Batteriebetrieb.}
	\item{Bedienungsanleitung für den allgemeinen Gebrauch des Prüfgerätes.}	
\end{itemize}

\section{Rahmenbedingungen}

\begin{itemize}
	\item{Abgabetermin Dokumentation IHK:	\tab	21.Mai 2021}	
	
	\item{Abgabetermin Prototyp BMK-Entwicklung:	\tab	14.Mai 2021}	
\end{itemize}


\vspace{1,5cm}
\begin{tabularx}{\textwidth}[b]{ p{5cm} } \cline{1-1} 
Ort, Datum
\end{tabularx}

\vspace{1,5cm}
\begin{tabularx}{\textwidth}[b]{ p{5cm} X p{5cm} } \cline{1-1} \cline{3-3} 
Unterschrift Auftraggeber & & Unterschrift Auftragnehmer
\end{tabularx}



\end{document}