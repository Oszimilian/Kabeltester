\documentclass[a4paper,11pt]{scrartcl}

\usepackage[utf8]{inputenc}
\usepackage[ngerman]{babel}
\usepackage[T1]{fontenc}
\usepackage{amsmath}
\usepackage{graphicx}
\usepackage{tabularx}
\usepackage[a4paper, left=2cm, right=2cm, top=2.8cm, bottom=2.8cm]{geometry}
\usepackage{tikz}   
\usepackage[scaled]{helvet}
\usepackage{tabto} 
\usepackage{fancyhdr}
\usepackage{multirow}

\renewcommand*{\familydefault}{\sfdefault}

\pagestyle{fancy}

\setkomafont{section}{\huge}
\setkomafont{subsection}{\Large}


\lhead{Maximilian Hoffmann}
\chead{Betrieblicher Auftrag \\ \textbf{Kabeltester}}
\rhead{\includegraphics[width=3cm]{Bilder/BMK_LOGO.png}}


%%%%%%%%%%%%%%%%%%%%%%%%%%%%%%%%%%%%%%%%%%%%%%%%%%%%%%%%%%%%%%%%%%%%%%%%%%%%%%%%%%%%%%%%%%%%%%%%%%%%%%%%%%%%%%%%%%%%%%%%%%%%%%%%%%%%%%%%%%%%%%%																																			 %
%														Ausgangssituation																     	 %
%																																		     %
%%%%%%%%%%%%%%%%%%%%%%%%%%%%%%%%%%%%%%%%%%%%%%%%%%%%%%%%%%%%%%%%%%%%%%%%%%%%%%%%%%%%%%%%%%%%%%%%%%%%%%%%%%%%%%%%%%%%%%%%%%%%%%%%%%%%%%%%%%%%%%
\begin{document}

\section{Ausgangssituation}

%%%%%%%%%%%%%%%%%%%%%%%%%%%%%%%%%%%%%%%%%%%%%%%%%%%%%%%%%%%%%%%%%%%%%%%%%%%%%%%
%																			  %
%								BMK-Entwicklung								  %
%																			  %
%%%%%%%%%%%%%%%%%%%%%%%%%%%%%%%%%%%%%%%%%%%%%%%%%%%%%%%%%%%%%%%%%%%%%%%%%%%%%%%

\subsection{BMK-Entwicklung}
Neben der Tätigkeit als EMS-Dienstleister (Electronics Manufacturing Services), bietet die \glqq BMK Group GmbH  Co. KG \grqq{} auch eine kundenspezifische Elektronikentwicklung an. Die Entwicklung von Software, Hardware und Layout erfolgt unter dem Namen\glqq BMK-Entwicklung \grqq{} . 
Zu meinen Aufgaben in der BMK-Entwicklung gehört es, den Hardwareentwicklern und Softwareentwicklern bei ihren täglichen Aufgaben bei Seite zu stehen. 

%%%%%%%%%%%%%%%%%%%%%%%%%%%%%%%%%%%%%%%%%%%%%%%%%%%%%%%%%%%%%%%%%%%%%%%%%%%%%%%
%																			  %
%								BMK-Testentwicklung							  %
%																			  %
%%%%%%%%%%%%%%%%%%%%%%%%%%%%%%%%%%%%%%%%%%%%%%%%%%%%%%%%%%%%%%%%%%%%%%%%%%%%%%%

\subsection{BMK-Testentwicklung / Testentwicklungslabor}
Eine weitere Dienstleistung der \glqq BMK Group \grqq{} ist die \glqq Testentwicklung \grqq{}. Im Haus gefertigte Baugruppen, durchlaufen während des Fertigungsprozesses verschiedene Testschritte. Die Entwicklung von Testadaptern, wird auf Kundenwunsch von der BMK-Testentwicklung übernommen. Standardisierte Testverfahren hierbei sind: ICT-Test (Integrated Circuit Test), FKT-Test (Funktionstest) und Boundary-Scan. Der Aufbau eines Testadapters wird dabei durch das Testentwicklungslabor übernommen. 

%%%%%%%%%%%%%%%%%%%%%%%%%%%%%%%%%%%%%%%%%%%%%%%%%%%%%%%%%%%%%%%%%%%%%%%%%%%%%%%
%																			  %
%								Testverfahren Akutell						  %
%																			  %
%%%%%%%%%%%%%%%%%%%%%%%%%%%%%%%%%%%%%%%%%%%%%%%%%%%%%%%%%%%%%%%%%%%%%%%%%%%%%%%

\subsection{Aktuelles Testverfahren für Programmierkabel sowie Flachbandkabel}
Das derzeitige Testverfahren für Programmierkabel in der Softwareentwicklung erfolgt nach dem\glqq Try and Error – Prinzip \grqq{} . Kann der Softwareentwickler keine Verbindung mit den gefertigten Programmierkabeln zwischen Controller und Programer aufbauen, so muss das Problem zeitaufwendig analysiert werden.
\\
Ist ein Testadapter des Testentwicklungslabors fertiggestellt, so wird dieser abschließend durch einen anderen Mitarbeiter überprüft. Dabei wird jede einzelne Leitung mit Hilfe eines Multimeters anhand des Schaltplanes überprüft. Flachbandkabel, welche Adapterkassette und Nadelbett miteinander verbinden, mussten sich bisher keinen Prüfungsprozess unterziehen. 

\end{document}