\section{Fehleranalyse}

\begin{center}
In diesem Kapitel werden die Fehler, welche bei der Inbetriebnahme aufgedeckt wurden behandelt. Der sich im Anhang befindenden Schaltplan \glqq Abschlussprojekt Kabeltester  V-01\grqq{} vermerkt diese mit der Beschriftung \glqq Fehler 1-5\grqq{}. Schaltungsdesignefehler wurden mit Fädeldraht verbessert.  
\end{center}



\subsection{Fehler 1}

Bei dem ersten Fehler handelt es sich um ein Footprintfehler. Der Sicherung F1 wurde im Schaltungsteil \glqq Versorgung \grqq{} auf Seite 2, ein Falsches Footprint zugewiesen. Somit ist es nicht möglich den Sicherungssockel zu verlöten. 


\subsection{Fehler 2}

Im Schaltungsteil \glqq NE555 Taktgeber \grqq{} auf Seite 3, wurde im Abschnitt \glqq Taktteilung \grqq{} das IC U8A (Inverter) verpolt. Durch anheben der Pin's 3 und 4 wurde mit der Hilfe von Fädeldraht dieser Fehler behoben.



\subsection{Fehler 3}

Bei der Inbetriebnahme der Konstantstromquelle auf Seite 7 stellte sich heraus, dass die Schaltung in dieser Konstellation zum schwingen beginnt. Durch einen Beitrag in einem Forum wurde ich auf einen fehlenden Widerstand zwischen TP3 und Pin 7 (U19A) hingewiesen. Mir wahr aus vorangegangenen Simulationen klar, dass der Kondensator C22 zum unterdrücken von Schwingungen benötigt wird. Der durch hohe Frequenzen auftretende Impedanzeinbruch, welcher die Ausgänge U19A Pin 7 und U17A Pin 4 miteinander kurzschließt, wurde mir in früheren Simulationen aus irgend einen Grund nicht Angezeigt und von meiner Seite aus nicht bedacht. Zur Fehlerbehebung wurde die Leiterbahn zwischen TP3 und Pin 7 (U19A) aufgetrennt und mit einem 10k Widerstand versehen. 



\subsection{Fehler 4}

Nach Behebung des \glqq Fehler 3 \grqq{} war ein korrektes einstellen des Konstantstromes nicht möglich. Das Problem hierbei liegt in der Versorgung von U19. Bei einem Idealen Leitungswiderstand von 0R liegt an Pin 4 die Versorgungsspannung von U19 an. Intern wird dieses Signal durch einen \glqq Nicht invertierenden Verstärker \grqq{} verstärkt. Da ein OP das Eingangssignal nie auf einen Pegel, größer als seine Betriebsspannung verstärken kann und durch Spannungsabfall am \glqq High und Low - Side Transistor \grqq{} der Ausgangsstufe das verstärkte Signal nicht den Wert der Versorgungsspannung erreichen kann, kommt es zu einer falschen Strommessung. Durch Trennung der Leitung, welche U19 mit +5V versorgt und anschließender Versorgung von U19 mit der Eingangsspannung (+9V) wurde dieses Problem behoben. 

\newpage

\subsection{Fehler 5}

Das zu erwartende Signal an TP36 konnte nicht gemessen werden. Der Schiebeimpuls kam dabei zu früh. Durch invertieren des \glqq System Takt 2 \grqq{} wurde dieser Fehler behoben.



\subsection{Fehler 6}

\textcolor{red}{
Wird der Eingang \glqq SRCLR \grqq{} der Schieberegister U20A und U21A mit dem Label \glqq RESET \grqq{} verbunden, so wird die Schaltung dauerhaft zurückgesetzt. 
Wird eine Messung gestartet, so kommt es zu einen positiven Nadelimpuls. Die \glqq Clear-Eingänge \grqq{} der Schieberegister sind dabei Low-Aktiv. 
\\
Durch eine Invertierung des RESET-Impulses oder durch das Verbinden mit +5V kann dieser Fehler behoben oder unterdrückt werden.
}
