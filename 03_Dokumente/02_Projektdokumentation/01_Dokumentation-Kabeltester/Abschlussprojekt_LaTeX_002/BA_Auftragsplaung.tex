\section{Auftragsplanung}

Ursprünglich war das Einsatzumfeld meines betrieblichen Auftrages für die BMK-Entwicklung vorgesehen. Während der Erstellung eines Lastenheftes stellte sich heraus, dass auch das Testentwicklungslobor an einem solchen Prüfgerät Interesse hat. 



\subsection{Auftragsziel}

Ziel des betrieblichen Auftrages ist die Entwicklung eines Prüfkonzeptes, sowie den dazugehörigen Prototypen einer Prüfvorrichtung zum Prüfen verschiedener konfektionierter Kabel. Das Prüfkonzept dieses Prototyps soll im Endzustand eine Prüfung verschiedener BMK-Standard-Programmierkabeln und Flachbandkabeln übernehmen. 
In Zukunft soll dadurch eine Menge Zeit bei der Fehlersuche eingespart werden und die Übergabe von fehlerhaften Programmierkabeln an einen Softwareentwickler ausgeschlossen werden.
Das Prüfgerät soll nach dem „stand alone“ -Prinzip funktionieren und für den Anwender einfach zu bedienen sein.


\subsection{Anforderungen}

Durch Rücksprache mit dem Auftraggeber wurden folgende Details ausgearbeitet. Diese wurden außerdem in einem offiziellen Pflichtenheft festgehalten. 

\begin{itemize}
	\item{Die extern angelegte Versorgungsspannung (Akku oder Batterie) soll möglichst verlustfrei auf einen 5V TTL-Pegel gewandelt werden. }

	\item{Die Leitungen sollen auf Durchgängigkeit, sowie auf Kurzschlüsse untereinander geprüft werden.}
	
	\item{Mit Hilfe eines konstanten Stromfluss, soll der ohmsche Widerstand jeder einzelnen Ader anhand eines festen Vergleichswertes überprüft werden.}
	
	\item{Anzahl der sich auf einer Prüfvorrichtung befindenden Messlinien: 14 Stück.
Erster Prototyp ist auf 4 Messlinien begrenzt!}

	\item{Einfache Erweiterung der Messlinien durch Verbindung zweier Prüfvorrichtungen.}
\end{itemize}


\subsection{Teilaufträge}
Um das Projekt im Rahmen des Zeitplanes voranzutreiben, werde ich im Bereich Layout und Bestückung der Platinen von meiner Abteilung unterstützt. Diesen Teilbereichen widme ich in meiner Dokumentation keine Aufmerksamkeit. Es wird lediglich die Anforderung an den jeweiligen Teilbereich schriftlich festgehalten. 

\subsection{Bauteilbeschaffung}
Die für mein Abschlussprojekt benötigten Bauteile sind größten Teils bei BMK im Haus vorrätig. Vereinzelte Bauteile, welche entweder nicht vorrätig oder durch \glqq Kundenspezifische Aufträge \grqq reserviert sind, mussten extern bestellt werden. Eine genaue Berechnung der Kosten, ist nicht möglich, da BMK bei verschiedenen Distributoren Rabatte, sowie spezielle Versandarten erhält und verwendet. Die Leiterplatten wurden bei einem chinesischen Leiterplattenhersteller produziert. 

\newpage

\textbf{Extern bestellte Bauteile:} 

\begin{itemize}
	\item{DG2535}
	
	\item{CD74HC4514M96}
	
	\item{SN74LS93}
\end{itemize}

\renewcommand{\arraystretch}{2}
\subsection{Zeitplanung}

\begin{tabularx}{\textwidth}{p{0.7\textwidth}|  p{0.1\textwidth} | p{0.2\textwidth}}

\textbf{Arbeitsschritt} 	&	\textbf{Geplant}	&	\textbf{Benötigt}	\\

\hline

Gespräch mit Auftraggeber über das Lastenheft	&	1	&	2\\

\hline

Erstes technisches Konzept erstellen			&	2	&	6\\

\hline

Erstellen eines Pflichtenheft					&	1	&	2\\

\hline 

Technische Planung der Schaltungsteile			&	4	&	8\\

\hline 

Bauteilrecherche  								&	4	&	6\\

\hline
\hline

Erstellen eines Blockschaltplanes				&	2	&	4\\

\hline

Zeichen der Einzelnen Schaltplanteile			&	9	&	16\\

\hline 

Bestimmen der Bauteilreferenzen					&	1	&	3\\

\hline

Übergabe an Layoutabteilung						&	0,5	&	0,5\\

\hline

Übergabe an Bestückung							&	0,5	&	0,5\\

\hline
\hline

Inbetriebnahme									&	2	&	4\\

\hline

Fehleranalyse und Fehlerbehebung				&	3	&	12\\

\hline 

Dokumentation									&	5	&	16\\

\hline

Übergabe an Auftraggeber						&	1	&	1\\

%\vspace{1cm}
\hline
\hline

\textbf{Gesamt}	&	36h		&	81h

\end{tabularx}
\renewcommand{\arraystretch}{1}